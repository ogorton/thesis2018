\chapter{CONCLUSION AND FUTURE WORK}
\label{chap:conclusion}

This thesis was a study of proton-neutron entanglement entropy and the 
presentation of a new interaction shell model code, PNISM. It was shown 
that the proton-neutron entanglement entropy is inversely correlated
with the isospin of a atomic nuclei. 
This hypothesis was further supported by a decomposition of nuclear 
wavefunctions into pure proton and pure neutron eigenstates, which 
fall off exponentially. The fact that nuclei with greater isospin
appear to have weaker proton-neutron coupling suggests that heavier
nuclei with high isospin could be computed with even greater efficiency 
using this truncation scheme. 

We wrote a J-scheme interacting shell model code which takes proton wavefunctions
and neutron wavefunctions and creates a truncated basis of coupled proton
and neutron wavefunctions. The nuclear Hamiltonian is computed in this basis
and solved using standard numerical techniques. The ground state energy 
and wavefunctions converge exponentially to those of the full model space.
The results for three nuclei in the ($p_{1/2}$, $p_{3/2}$, $f_{5/2}$, $f_{7/2}$) model space, which would normally 
require a supercomputer to solve, were computed using PNISM running on 
a laptop. Results are not satisfactory and further research will need to be done to 
improve convergence. Eventually we would like be able to compute nuclei with full model
space dimensions in the billions using truncated model spaces in 
PNISM; however we are still working on some bugs to do with 
model spaces with more than one parity. 

In the future we will need to make changes to the code to improve convergence.
The model presented here is the simplest version in that basis states are
selected from the lowest eigenstates of the proton and neutron Hamiltonians.
This could be improved by introducing effective single-particle energies
to account for the contribution from the remaining states which were left out,
perhaps through some mean-field approximation.

In the current version, PNISM computes the entire set of Hamiltonian matrix
elements before diagonalizing. Memory utilization could be greatly improved
by diagonalizing the Hamiltonian after completing each value of total angular
momentum. This is possible since PNISM is a J-scheme code; matrix elements between states with
different total $J$ are independent. Here is a pseudo-code depiction of the current
procedure:
\begin{verbatim}
    DO J = JMIN, JMAX
        COMPUTE <AB,J| H |CD,J>
    END DO
    DO J = JMIN, JMAX
        DIAGONALIZE <AB,J| H |CD,J>
    END DO
    DEALLOCATE <AB,J| H |CD,J>
\end{verbatim}
The problem being that this requires storing the entire Hamiltonian in memory
before diagonalizing and writing to disk the much smaller set of truncated 
eigenstates and quantum numbers. If instead we use the following procedure:
\begin{verbatim}
    DO J = JMIN, JMAX
        COMPUTE <AB,J| H |CD,J>
        DIAGONALIZE <AB,J| H |CD,J>
        DEALLOCATE <AB,J| H |CD,J>
    END DO
\end{verbatim}
then we can expect a reduction in the total memory resources required by 
a factor that scales like the number of $J$ states required. While simple 
in principle, this will require a nontrivial restructuring of the 
code.

This reorganization of the code would also be compatible with a hybrid 
implementation of MPI\cite{mpi} and openMP\cite{openmp}. MPI is a 
distributed memory message passing interface commonly used for
parallelization. MPI will be called to distribute iterations
of the loop over $J$ values to various nodes of a cluster.
Each node, being assigned some allotment of work, will 
create numerous threads using openMP, a shared memory
multi-processing programming interface. If only one node is
available, then creation of openMP threads will be preferred.
