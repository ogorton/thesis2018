\chapter{PNISM}
\label{chap:PNISM}

% By labeling the chapter, I can refer to it later using the
% label. (\ref{chap:intro}, \pageref{chap:intro}) Latex will take care
% of the numbering.
Finally, we turn to the main subject of this thesis, our Proton-Neutron Interacting
Shell Model (PNISM) code. We are motivated by the results of the previous chapters; 
it is known that SVD eigenvalues of wavefunction coefficients fall off exponentially,
and we have seen that they fall off even faster in nuclei with unequal numbers
of protons and neutrons. We also argued that the basis states formed from
the eigenstates of the proton interaction and the neutron interaction could likely
allow for truncated wavefunctions. We have reason to believe that truncation of 
the model space in a weakly coupled proton-neutron basis might be an especially effective
approximation for heavier nuclei with more neutrons than protons. 
We set out to write a code that couples together eigenstates of the proton Hamiltonian and 
eigenstates of the neutron Hamiltonian to create a many-proton many-neutron basis.
The matrix elements of the full Hamiltonian are then computed in this basis, which is
truncated.

This chapter will derive the equations used to carry out the calculation of the
Hamiltonian matrix elements from the eigenstates computed by the interacting 
shell model code BIGSTICK. Calculation of the one-body density matrices of the
resulting wavefunctions is also discussed. The following chapter will discuss
the details of the code itself.

\section{The Hamiltonian}
We write the nuclear Hamiltonian in explicit proton-neutron formalism. The 
Hamiltonian operator in the second quantization formalism,
\begin{equation}
	\hat{H}=\sum_{ab}\bra{a}T\ket{b}\hat{c}^\dagger_a\hat{c}_b
    +\frac{1}{4}\sum_{abcd}\bra{ab}V\ket{cd}\hat{c}^\dagger_a\hat{c}^\dagger_b\hat{c}_c\hat{c}_d,
\end{equation}
can be written as a sum of terms separated by nucleon species\cite{Johnson13}:
\begin{equation}\label{sep}
    \hat{H} = \hat{H}_p + \hat{H}_n + \hat{H}_{pp} + \hat{H}_{nn} + \hat{H}_{pn},
\end{equation}
where $\hat{H}_p$ and $\hat{H}_{pp}$ are one- and two-body Hamiltonians, respectively, which act
only on protons, $\hat{H}_n$ and $\hat{H}_{nn}$ act only on neutrons, and $\hat{H}_{pn}$
contains the remaining interactions between proton and neutrons. 
To this end, it is useful to define the two body creation operator\cite{Schuck}:
\begin{equation}
    \hat{A}_{JM}^\dagger(ab)\equiv [c_a^\dagger \otimes c_b ^\dagger]_{JM}
\end{equation}
Here, the total angular momentum $J$ and the azimuthal or magnetic quantum number 
$M=J_z$ are needed for the coupling of the two one-body creation operators, 
via Clebsch-Gordan coefficients\cite{Edmonds}:
\begin{equation}
    [c_a^\dagger \otimes c_b ^\dagger]_{JM} = 
        \sum_{m_am_b} (j_am_a, j_bm_b|JM)\hat{c}^\dagger_{j_am_a}
        \hat{c}^\dagger_{j_bm_b}.
\end{equation}
The two-body creation and annihilation operators\cite{Feshbach} with fixed $J$ and $M$ are thus, respectively:
\begin{equation}\begin{split}\label{pairs}
    & A^\dagger_{JM} = \sum_{m_am_b} (j_a m_a, j_b m_b | JM ) c^\dagger_{j_am_a} 
c^\dagger_{j_bm_b}, \textrm{ and} \\
    & A_{JM} = \sum_{m_cm_d} (j_c m_c, j_d m_d | JM ) c_{j_cm_c} c_{j_dm_d}.
\end{split}\end{equation}
These will be used later in the derivation of the proton-neutron interaction
matrix elements.
With these definitions, two-body states with proper normalization are\cite{Brussard}
\begin{equation}
    \ket{ab; JM} \equiv \frac{1}{\sqrt{1+\delta_{ab}}}
       \hat{A}^\dagger_{JM}(ab)\ket{0}.
\end{equation}
The two-body part of the Hamiltonian is thus expressed
\begin{equation}\begin{split}
    \hat{H_{2b}} &= \sum_{abcd}{c}^\dagger_a{c}^\dagger_b\bra{ab}V\ket{cd}
        \hat{c}_c\hat{c}_d \\
    &= \frac{1}{4}\sum_{abcd}\zeta_{ab}\zeta_{cd}
    \sum_{J}\bra{ab;JM}V\ket{cd;JM}\sum_{M}{A}^\dagger_{JM}(ab){A}_{JM}(cd)
\end{split}\end{equation}
where $\zeta_{ab}\equiv\sqrt{(1+\delta_{ab})}$.
Our next move is to change to an explicit proton-neutron decomposition.
This can be done by using the states $\frac{1}{\sqrt{2}}
(\ket{\pi\nu}-\ket{\nu\pi})$, $\frac{1}{2}(\ket{\pi\nu}+\ket{\nu\pi})$,$\ket{\pi\pi}
$, and $\ket{\nu\nu}$ to isolate operators by the nucleon species on which they act. 
The result is the following proton-proton two-body interaction:
\begin{equation}\begin{split}
    \hat{H}_{pp} =
    \frac{1}{4}\sum_{abcd}\zeta_{ab}\zeta_{cd}
    \sum_{J}V_{J}^{pp}(ab,cd)\sum_{M}\hat{A}^\dagger_{JM}(a_\pi b_\pi)\hat{A}_{JM}(c_\pi d_\pi).
\end{split}\end{equation}
$\hat{H}_{nn}$ has the same form. We will not use these directly but it is 
useful to see how the interaction is separated. The proton-neutron interaction is
the only term that will be used directly in our code:
\begin{equation} \label{eqn: hpn}
    H_{pn} = \sum_{abcd}\sum_J V_J^{(pn)}(ab,cd) 
\sum_M A_{JM}^\dagger(a_\pi b_\nu) A_{JM}(c_\pi d_\nu).
\end{equation}


\section{The Basis}
We have a Hamiltonian of the form (\ref{sep}) and in order to compute its matrix
elements, we need to choose a basis, one which we hope we can truncate effectively. In this
section I will explain our choice of basis and how we select which states to truncate.
In the future we will want to develop a more sophisticated approximation scheme to 
improve our results. 
For now we first solve
\begin{equation}\begin{split}\label{ppnnsol}
    (\hat{H}_p + \hat{H}_{pp})\ket{\Psi_p}&= E_p\ket{\Psi_p} \textrm{ and}\\
    (\hat{H}_n+H_{nn})\ket{\Psi_n}&=E_n\ket{\Psi_n},
\end{split}\end{equation}
and set our uncoupled basis states equal to the solutions:
\begin{equation}\begin{split}\label{plusminus}
    \ket{J_p^{\pi_p},\alpha_p}&=\ket{\Psi_p},\\
    \ket{J_n^{\pi_n},\beta_n}&=\ket{\Psi_n},
\end{split}\end{equation}
where $J_p$ is the total angular momentum of the proton state, $\pi_p$ is its parity,
and $\alpha_p$ is just a state label, and equivalently for the neutron states. 
We build our many-body basis states by coupling together these proton and
neutron states: 
\begin{equation}\label{cbasis}
	\ket{J_p^{\pi_p},\alpha_p}\otimes\ket{J_n^{\pi_n},\beta_n}.
\end{equation}

Our
solutions to (\ref{ppnnsol}) yield $d_\pi$ pure proton eigenstates and $d_\pi$ pure
neutron eigenstates. The eigenstates of a Hermitian operator form a complete set 
of orthogonal eigenvectors. Assuming these are normalized, any many-body 
wavefunction can be expanded:
\begin{equation}
    \ket{\Psi} = \sum_{\alpha_p}^{d_\pi}\sum_{\beta_n}^{d_\nu}\Psi_{\alpha_p\beta_n} \ket{J_p^{\pi_p},\alpha_p}\otimes\ket{J_n^{\pi_n},\beta_n}.
\end{equation}
Therefore these states can be used to compute the matrix elements of the full Hamiltonian
(\ref{sep}), where $\hat{H}_p + \hat{H}_{pp}$ and $\hat{H}_n+H_{nn}$ are now
used in their diagonal forms.

So far all we have done is to solve the nuclear Hamiltonian in a basis of
coupled proton and neutron states by solving the proton and neutron 
interactions separately. The computational advantage to this
procedure lies in the prospect of truncating the model space. In doing so, we 
will have to make choices to ensure that the eigenstates of the truncated
Hamiltonian remain eigenstates of $J^2$. 
If we were to truncate the basis in an M-Scheme procedure, then our
solutions would not be guaranteed to have good total angular momentum. To see why,
consider the following example for a two state system. Imagine a basis with two
states $\frac{1}{\sqrt{2}}(\ket{z;+}+\ket{z;-})$ and
$\frac{1}{\sqrt{2}}(\ket{z;+}-\ket{z;-})$. If we were
truncate to just one of either of these states, then any solution in the truncated
basis could not be an eigenstate of $\hat{S}_z$. Similarly, a truncated M-scheme
basis is not guaranteed to have good total angular momentum. Therefore,
we choose a basis with good total angular momentum so that even once we
truncate the basis we are guaranteed to get solutions with good total angular momentum.
This is called a J-scheme.

Therefore we choose to couple our proton and neutron states in such a way as to
be eigenstates of $\hat{J}^2$:
\begin{equation}\label{cbasisJ}
	[\ket{J_p^{\pi_p},\alpha_p}\otimes\ket{J_n^{\pi_n},\beta_n}]_{J^\pi}\equiv
	\ket{J_p^{\pi_p},\alpha_p,J_n^{\pi_n},\beta_n;J^\pi},
\end{equation}
A wavefunction expanded in this basis is of the form
\begin{equation}
    \ket{\Psi} = \sum_{\alpha_p\beta_n}^{Q}\Psi_{\alpha_p\beta_n} 
    \ket{J_p^{\pi_p},\alpha_p,J_n^{\pi_n},\beta_n;J^\pi},
\end{equation}
and is guaranteed to have good total $J$. Here, the sum runs up to $Q$, which
will be chosen to be $Q << min[d_p,d_n]$ when computing the matrix elements of 
(\ref{sep}).





\section{Factorizing the Proton-Neutron Interaction}

To calculate the matrix elements of $H_{pn}$, we refactor equation (\ref{eqn: hpn})
into one-body density-like operators whose matrix elements can be computed from the density 
matrices found from solving the single-species terms (see equation (\ref{ppnnsol}))
of the full Hamiltonian (\ref{sep}). Equation (\ref{eqn: hpn}) is written
as a product of pair creation and annihilation operators. These can be reordered
into one-body density operators. Doing so will introduce a nontrivial phase and
a number of identities involving vector coupling coefficients will be applied.


Expanding equation (\ref{eqn: hpn}) using (\ref{pairs}), we obtain
\begin{equation}\begin{split}\label{hpnexpanded}
    H_{pn} = \sum_{abcd}&\sum_J V_J^{(pn)}(ab,cd) \\
    &\sum_M \sum_{m_am_b}\sum_{m_cm_d} (j_a m_a, j_b m_b | JM )
        (j_c m_c, j_d m_d | JM )c^\dagger_{j_am_a}c^\dagger_{j_bm_b}c_{j_cm_c} c_{j_dm_d},
\end{split}\end{equation}
where the sums are constrained such that $M=m_a+m_b=m_c+m_d$. We want to write (\ref{hpnexpanded})
in terms of the the generalized one-body density operator:\cite{Schuck}:
\begin{equation}
    \rho_{K\mu}(a\tilde{c}) \equiv \sum_{m_a m_c} (j_a m_a, j_c -m_c | K\mu) 
c^\dagger_{j_am_a} \tilde{c}_{j_c-m_c}.
\end{equation}
Given the definition of a time-reversed annihilation operator\cite{Edmonds},
\begin{equation} 
    \tilde{c}_{j_c-m_c} = (-1)^{j_c-m_c}c_{j_c m_c},
\end{equation}
we write the generalized one-body density operator in terms of the same creation/annihilation
operators that appear in (\ref{hpnexpanded}):
\begin{equation}\label{rhokm}
    \rho_{K\mu}(a\tilde{c}) = \sum_{m_a m_c} (j_a m_a, j_c -m_c | K\mu)
(-1)^{j_c -m_c} c^\dagger_{j_a m_a} c_{j_c m_c}.
\end{equation}
The tilde over the index $c$ reminds us the time reversal 
operator was used. To move further we need to
recouple the vector coupling coefficients in (\ref{hpnexpanded}) to match
the coefficients in the definition (\ref{rhokm}). This requires \textit{six-J symbols}\cite{Edmonds},
an invariant generalization of vector coupling coefficients. A brief introduction 
to vector coupling coefficients and six-J symbols are given in Appendix 
\ref{appendix: vc}. We will need the following relationship
between vector-coupling coefficients and the six-J symbol\cite{Edmonds}:
\begin{equation}\begin{split}\label{edmondsid1}
    (j_a m_a, j_b m_b | JM )&(j_c m_c, j_d m_d | JM )= \\ 
\sum_{K\mu} (-1)^{J+M}&(-1)^{j_b-j_d+\mu}(2J+1) \begin{Bmatrix} j_a & j_b & J \\ 
j_d & j_c & K \end{Bmatrix} \\ 
    &(j_a m_a, j_c-m_c | K\mu )(j_b m_b, j_d -m_d | K-\mu ).
\end{split}\end{equation}
Now that we have what we need, we start again with 
(\ref{eqn: hpn}) and sequentially apply these definitions and
identities. The pair creation and annihilation operators in 
(\ref{eqn: hpn}) can be rewritten using (\ref{edmondsid1}):

\begin{equation}\begin{split}\label{derivation}
   \sum_M  A^\dagger_{JM}(ab)&\ A_{JM}(cd) \\
           =\sum_{M}\sum_{m_am_b}&\sum_{m_cm_d} (j_am_a,j_bm_b|JM)(j_cm_c,j_dm_d|JM)
                c^\dagger_{j_am_a}c^\dagger_{j_bm_b}c_{j_cm_c}c_{j_dm_d} \\
           =\sum_{M}\sum_{m_am_b}&\sum_{m_cm_d}\sum_{K\mu}(-1)^{J+M}(-1)^{j_b-j_d+\mu}(2J+1)
                \begin{Bmatrix} j_a & j_b & J \\ j_d & j_c & K \end{Bmatrix} \\ 
          & (j_a m_a, j_c-m_c | K\mu )(j_b m_b, j_d -m_d | K-\mu )
                c^\dagger_{j_am_a}c^\dagger_{j_bm_b}c_{j_cm_c}c_{j_dm_d}
\end{split}\end{equation}
We apply an anticommutation relation to reorder the creation/annihilation operators:
\begin{equation}\begin{split}\label{usecom}
     c^\dagger_{j_am_a}c^\dagger_{j_bm_b}c_{j_cm_c}c_{j_dm_d} &= 
        c^\dagger_{j_am_a}(\delta_{j_bm_b,j_cm_c}-c_{j_cm_c}c^\dagger_{j_bm_b})c_{j_dm_d} \\
        &=\delta_{j_bj_c}\delta_{m_bm_c}c^\dagger_{j_am_a}c_{j_dm_d} 
        - c^\dagger_{j_am_a}c_{j_cm_c}c^\dagger_{j_bm_b}c_{j_dm_d},
\end{split}\end{equation}
where the term containing $c^\dagger_{j_am_a}c_{j_dm_d}$ is a charge-changing 
operator and can be ignored. Inserting (\ref{usecom}) into (\ref{derivation})
and applying the definition (\ref{rhokm}), 
\begin{equation}\begin{split}
    &\sum_M A^\dagger_{JM}(ab)\ A_{JM}(cd) \\
    &=\sum_M\sum_{K\mu}(-1)^{J+M+j_b-j_d+\mu}(2J+1)
        \begin{Bmatrix} j_a & j_b & J \\ j_d & j_c & K \end{Bmatrix}
    (-1)^{-j_c+m_c-j_d+m_d}\rho_{K\mu}
                (a\tilde{c})\rho_{K-\mu}(b\tilde{d}) \\
          &=\sum_{K\mu}(-1)^{J+j_b+j_c}(2J+1)\begin{Bmatrix} j_a & j_b & J \\ 
                j_d & j_c & K \end{Bmatrix}(-1)^\mu \rho_{K\mu}(a\tilde{c})\rho_{K-\mu}
                (b\tilde{d}),
\end{split}\end{equation}
where I applied the condition that $M=m_a+m_b=m_c+m_d$ and the general property that 
$(-1)^{j} = (-1)^{j+2n}$ when $n$ is an integer. In the second equality the 
constraint  $\mu=m_a-m_c=-(m_b-m_d)$ eliminates the sum over $M$. To further simplify
our notation, we define the dot product:
\begin{equation}\begin{split}
	\rho_{K}(a\tilde{c}) \cdot \rho_{K}(b\tilde{d}) &= \sum_\mu (-1)^K
(2K+1)^{-1/2}(K\mu,K-\mu|00)\rho_{K\mu}(a\tilde{c})\rho_{K-\mu}(b\tilde{d}) \\
	&= \sum_{\mu} (-1)^{-\mu} \rho_{K\mu}(a\tilde{c})\rho_{K-\mu}(b\tilde{d}),
\end{split}\end{equation}
where we applied the following identity for the case where we are coupling 
up to zero total angular momentum\cite{Edmonds}:
\begin{equation}\label{couplezero}
	(K\mu,K-\mu|00) = (-1)^{K-\mu}(2K+1)^{-1/2}.
\end{equation} 
Thus the proton-neutron Hamiltonian can be written as
\begin{equation}\begin{split}
	H_{pn} = \sum_{abcd}\sum_J V_J^{(pn)}(ab,cd)\sum_{K}(-1)^{J+j_b+j_c}(2J+1)
\begin{Bmatrix} j_a & j_b & J \\ j_d & j_c & K \end{Bmatrix}\rho_{K}(a\tilde{c}) 
\cdot \rho_{K}(b\tilde{d}),
\end{split}\end{equation}
a coupling of the proton and neutron density matrices through a  
factor we define as:
\begin{equation} \label{eqn: wmat}
	W_K(ac,bd) \equiv \sum_J (-1)^{J+j_b+j_c}(2J+1)\begin{Bmatrix} j_a & j_b & J \\ 
j_d & j_c & K \end{Bmatrix}V_J^{(pn)}(ab,cd).
\end{equation}
Our Hamiltonian is now in a form that can be recognized as potential matrix elements
$W_K(ac,bd)$ times one-body density operators $\hat{\rho}_{K}(a\tilde{c}) \cdot \hat{\rho}_{K}
(b\tilde{d})$:
\begin{equation}\begin{split}\label{Hpnfull}
	H_{pn} = \sum_{abcd}\sum_{K}W_K(ac,bd)\hat{\rho}_{K}(a\tilde{c}) \cdot \hat{\rho}_{K} (b\tilde{d}).
\end{split}\end{equation}



\section{Matrix elements}
Matrix elements can now be computed in our basis of coupled proton-neutron states
using the above form of the Hamiltonian expressed in terms of density operators.
Our full Hamiltonian is 
\begin{equation}
 H = H_p + H_{pp} + H_n + H_{nn} + H_{pn}
\end{equation}
and the basis states are $\ket{J_p^{\pi_p},\alpha_p,J_n^{\pi_n},\beta_n|J^{\pi}}$. 

The proton-only and neutron-only terms have matrix elements defined as
\begin{equation}
    \bra{J_p,\pi_p,\alpha_p} H_p+H_{pp} \ket{J_p,\pi_p,\alpha_p} \equiv 
        h^{J_p^{\pi_p}}(\alpha_p',\alpha_p)
\end{equation}
and
\begin{equation}
    \bra{J_n,\pi_n,\alpha_n} H_n + H_{nn} \ket{J_n,\pi_n,\alpha_n} \equiv 
        h^{J_n^{\pi_n}}(\beta_n',\beta_n)
\end{equation}
These are the single species operators whose matrix elements will be found as
solutions from BIGSTICK.                                         
\begin{equation}\begin{split}
    &(H_p+H_{pp}) \ket{J_p,\pi_p,\alpha_p} = E(\alpha_p)\ket{J_p,\pi_p,\alpha_p} \textnormal{ and,} \\
    &(H_n+H_{nn}) \ket{J_n,\pi_n,\alpha_n} = E(\alpha_n)\ket{J_n,\pi_n,\alpha_n}.
\end{split}\end{equation}
Thus we have:
\begin{equation}\begin{split}
    h^{J_p^{\pi_p}} &= \delta_{a_p'a_p} E(\alpha_p) \textnormal{ and,} \\
    h^{J_n^{\pi_n}} &= \delta_{b_n'b_n}E(\alpha_n),
\end{split}\end{equation}
which contribute:
\begin{equation}
    \delta_{J_n'J_n}\delta_{J_p'J_p}\Big( \delta_{\beta_n'\beta_n}h^{J_p}
(\alpha_p',\alpha_p) + 
 \delta_{\alpha_p'\alpha_p}h^{J_n}(\beta_n'\beta_n)\Big). 
\end{equation}

The final term $H_{pn}$ is a weighted sum of scalar products between two 
tensor operators of the form:
\begin{equation}\label{tpd}
    T(k)\cdot U(k) = \rho_{K}(a\tilde{c}) \cdot \rho_{K} (b\tilde{d}).
\end{equation}
(See equation (\ref{Hpnfull}.) Since $\rho_K(a\tilde{c})$ acts only on protons,
and $\rho_K(b\tilde{d})$ acts only on neutrons, these two operators
commute, and can be computed in a coupled basis using a reduction to a product of 
operators in the uncoupled basis\cite{Edmonds}:
\begin{equation}\begin{split}
    \bra{j_1'j_2';J'M'}&T(k)\cdot U(k)\ket{j_1j_2;JM} = \\
        &(-1)^{j_1+j_2'+J}\delta_{J'J}\delta_{M'M}
\begin{Bmatrix} 
J & j_2' & j_1' \\ 
k & j_1 & j_2 
\end{Bmatrix}   \bra{j_1'}T(k)\ket{j_1}\bra{j_2'}U(k)\ket{j_2}
\end{split}\end{equation}
In the case of (\ref{tpd}), we will have our dependence on 
$\bra{J_p'^{\pi_p'},\alpha_p'}\rho_k(a\tilde{c})\ket{J_p^{\pi_p},\alpha_p}$ and
$\bra{J_n'^{\pi_n'},\beta_n'}\rho_k(b\tilde{d})\ket{J_n^{\pi_n},\beta_n}$.
These are just the proton and neutron density matrices represented in the basis 
of uncoupled proton and neutron states, respectively. Recalling that these 
uncoupled states are the eigenstates of the pure proton and pure neutron 
parts of the full Hamiltonian, these density matrices are 
density matrices produced by BIGSTICK when computing
$H_p + H_{pp}$ and $H_n + H_{nn}$.

Then the proton-neutron Hamiltonian matrix elements are
\begin{equation}\begin{split}\label{eqn: hmat}
\bra{J_p'^{\pi_p'},\alpha_p',J_n'^{\pi_n'},\beta_n';J^{\pi}}
 &\hat{H}_{pn} 
 \ket{J_p^{\pi_p},\alpha_p,J_n^{\pi_n},\beta_n;J^{\pi}} = \\
\sum_{ac,bd}\sum_{K}& W_K(ac,bd)(-1)^{J_p+J_n'+J}\begin{Bmatrix} J & J_n' & J_p' \\ K & J_p & J_n \end{Bmatrix} \\
&\bra{J_p'^{\pi_p'},\alpha_p'}\rho_K(a\tilde{c})\ket{J_p^{\pi_p},\alpha_p}
\bra{J_n'^{\pi_n'},\beta_n'}\rho_K(b\tilde{d})\ket{J_n^{\pi_n},\beta_n}
\end{split}\end{equation}
This matrix is then solved using techniques from linear algebra. In PNISM, the 
user can either select the included Lanczos algorithm to solve for low lying 
states or to use the full Householder diagonalization option using
the routine SSYEV from the LAPACK\cite{LAPACK} library, at the cost of increased run time.

The next major step is to calculate the one-body density matrices,  
used to calculate transition strengths. This is addressed in the next section.


\section{One-body density matrices}
A basic functionality of any configuration interaction code is the ability to 
calculate density matrices.
A general one-body density matrix is defined by
\begin{equation}\label{eq.densitymat}
	\rho_k^{f,i}(ab)=\bra{\Psi_f}[c_a^{\dagger}\otimes c_b]_k\ket{\Psi_i},
\end{equation}
which finds its use in the calculation of operator matrix elements
\begin{equation}
	\bra{\Psi_f}\hat{O}\ket{\Psi_i}=\sum_{ab} \rho_k^{f,i}(ab) \bra{a}\hat{O}\ket{b}.
\end{equation}

In order to calculate the density matrix elements between the eigenstates of the proton-
neutron Hamiltonian,
which are output from PNISM in the basis of coupled proton and neutron wavefunctions,
we need to be able to write (\ref{eq.densitymat}) in terms of the density matrices 
provided by BIGSTICK.
These, however, are computed in the basis of pure proton and pure neutron wavefunctions.
This section walks through the derivation of the one-body density matrix elements for 
our eigenstates
in terms of the one-body density matrices from the donor wavefunctions.

The matrix elements of a tensor product operator in a basis of coupled states 
are\cite{Edmonds}
\begin{equation} \begin{split}
\bra{j_1'j_2';J'}[\hat{T}_{k_1}\otimes \hat{U}_{k_2}]_K\ket{j_1j_2;J} \\
=[J'][K][J]
\begin{Bmatrix} 
j_1' & j_1 & k_1 \\ 
j_2' & j_1 & k_2 \\
J' & J & K
\end{Bmatrix}
\bra{j_1'}\hat{T}_{k_1}\ket{j_1}\bra{j_2'}\hat{U}_{k_2}\ket{j_2}.
\end{split} \end{equation}
If either $\hat{T}_{k_1}$ or $\hat{U}_{k_2}$ is the identity operator, 
as is the case for the 
one-body density matrices, then this expression further simplifies to 
\begin{equation}\begin{split}
\bra{j_1'j_2';J'}[\hat{T}_{k_1}]\ket{j_1j_2;J} \\
= \delta_{j_2'j_2}(-1)^{j_1'+j_2+J+K}[J'][J]
\begin{Bmatrix} 
j_1' & J' & j_2 \\ 
J & j_1 & k
\end{Bmatrix}
\bra{j_1'}\hat{T}_k\ket{j_1}
\end{split}\end{equation}
and similarly for $\hat{U}_{k_2}$.

Thus we can find the proton density matrix elements in the coupled basis in terms of 
the density matrices in the uncoupled basis as
\begin{equation}\label{eq.tensor}\begin{split}
\bra{j_{\pi}'j_{\nu}';J'}[\pi_a^{\dagger}\otimes \pi_c]_K\ket{j_{\pi}j_{\nu};J} \\
= \delta_{j_{\nu}'j_{\nu}}(-1)^{j_{\pi}'+j_{\nu}+J+K}[J'][J]
\begin{Bmatrix} 
j_{\pi}' & J' & j_{\nu} \\ 
J & j_{\pi} & k
\end{Bmatrix}
\bra{j_{\pi}'}[\pi_a^{\dagger}\otimes \pi_c]_K\ket{j_{\pi}}
\end{split}\end{equation}
The Kronecker-$\delta$ from the identity operator on the neutron space, 
and the orthogonality of the basis states. If more quantum numbers were included,
more Kronecker-$\delta$s would have to be added. Similarly, for the neutron density 
operator $[\nu_b^{\dagger}\otimes \nu_d]_K$ we have

\begin{equation}\label{eq.tensor2}\begin{split}
\bra{j_{\pi}'j_{\nu}';J'}[\nu_b^{\dagger}\otimes \nu_d]_K\ket{j_{\pi}j_{\nu};J} \\
= \delta_{j_{\pi}'j_{\pi}}(-1)^{j_{\pi}+j_{\nu}+J'+K}[J'][J]
\begin{Bmatrix} 
j_{\nu}' & J' & j_{\pi} \\ 
J & j_{\nu} & k
\end{Bmatrix}
\bra{j_{\pi}'}[\pi_a^{\dagger}\otimes \pi_c]_K\ket{j_{\pi}}.
\end{split}
\end{equation}
Note that this is not as simple as exchanging $\pi$ and $\nu$.

Finally, to get the expression for the density matrix for our coupled-state solutions, recall
that our states $\Psi_f$ and $\Psi_i$ in (\ref{eq.densitymat}) are solutions to 
\begin{equation}
\hat{H}_{} \ket{\Psi} = E \ket{\Psi}
\end{equation}
and are given in the basis of coupled proton-neutron states. Each state can be written as
\begin{equation}\label{eq.decomp}
\ket{\Psi}=\sum_\alpha c_\alpha \ket{j_{\pi}j_{\nu};J}_\alpha.
\end{equation}
Combining this expression (\ref{eq.decomp}) and the relation (\ref{eq.tensor}) with our definition 
(\ref{eq.densitymat}), we obtain the desired result
\begin{equation}\begin{split}\label{eqn: denmat}
\rho_k^{f,i}(\pi_a\pi_c)=\bra{\Psi_f}[\pi_a^{\dagger} \pi_c]_k\ket{\Psi_i} \\
=\sum_{\alpha\beta}c_{\alpha}^fc_{\beta}^i 
\bra{j_{\pi}^\alpha j_{\nu}^\alpha;J^f}
[\pi_a^{\dagger} \pi_c]_k
\ket{j_{\pi}^\beta j_{\nu}^\beta;J^i} \\
=\sum_{\alpha\beta}c_{\alpha}^fc_{\beta}^i
\delta_{j_{\nu}^\alpha j_{\nu}^\beta}(-1)^{j_{\pi}^\alpha+j_{\nu}^\beta+J^i+K}[J^f][J^i]
\begin{Bmatrix} 
j_{\pi}^\alpha & J^f & j_{\nu}^\beta \\ 
J^i & j_{\pi}^\beta & k
\end{Bmatrix}
\bra{j_{\pi}^\alpha}[\pi_a^{\dagger} \pi_c]_K\ket{j_{\pi}^\beta}.
\end{split}\end{equation}
The neutron one-body density matrix has identical structure.

This concludes the analytic discussions necessary to write the proton
neutron interacting shell model code, PNISM. We have the expression for both the
Hamiltonian matrix elements in terms of quantities which can be obtained from 
solutions from BIGSTICK (or any properly formatted results from an interacting 
shell model code), and the one-body density matrices which are used to compute
transition rates. Although the code for computing one-body density matrices has 
been written and tested, we have yet to apply it. Equation (\ref{eqn: hmat}) along with (\ref{eqn: wmat}) and 
(\ref{eqn: denmat}) are the actual equations coded up in PNISM.







