\chapter{INTRODUCTION}
\label{chap:intro}

% By labeling the chapter, I can refer to it later using the
% label. (\ref{chap:intro}, \pageref{chap:intro}) Latex will take care
% of the numbering.

Nuclear physics is the study of the structure and properties of atomic nuclei, 
which make up most of the known matter in the universe.
The goal of nuclear theory is to make predictions about the interactions
of nuclei with external fields, in order to answer questions 
about experimental outcomes and thereby related deeper and ongoing questions in physics
about the fundamental symmetries of nature.

Some of the more important and open questions in physics, such as those related to 
the matter-antimatter symmetry violation, the source and origin of dark matter 
in the universe, and neutrinoless double beta decay, all rely on accurate and 
detailed quantum models of nuclei. 
Investigations 
into neutrinoless double beta decay call for transition matrix elements from 
$^{76}$Ge and $^{76}$Se\cite{Faessler,Brown2}. 
Some searches for dark matter involve collisions
with heavy nuclei such as $^{131}$Xe, and require detailed nuclear structure calculations\cite{Bednyakov}.
Finally, the origin of the matter-antimatter symmetry 
violation may be due to CP-violation\cite{Sakhorav}, and one approach to investigating the 
source of CP-violation is through the permanent electric dipole moments of certain
nuclei such as $^{199}$Hg\cite{Willmann}. 

In order to understand observations obtained 
from experiments, or to make predictions about future experiments, we need to be 
able to compute the matrix elements of operators with accurate nuclear 
wavefunctions computed from detailed Hamiltonians.
I will be addressing the task of computing accurate wavefunctions from existing
Hamiltonians, by solving the quantum N-body problem as a matrix
eigenvalue problem. Because the complexity of many body problems scales exponentially
with the number of particles, the dimension of these matrices can very large. This is
especially true for medium and heavy nuclei. As a particular
example, calculations of the parity violating nuclear ``anapole moment'' for nuclei such as $^{133}$Cs and 
$^{205}$Tl rely on accurate ground state wavefunctions, and one- and two-body 
density matrix elements\cite{Haxton89,Haxton02}. $^{133}$Cs in the configuration interaction shell model,
a kind of matrix eigenvalue problem, has a basis dimension 
of over one-hundred million and requires nearly $800$ GB of memory to store
just the non-zero matrix elements of the Hamiltonian. Given the enormity of 
such problems and finite computer memory, it becomes of interest 
to find ways to reduce the size of such matrices. In principle, this can be done by truncating
the basis and computing only the matrix elements whose basis elements are most important.
In this thesis I will present in considerable detail evidence that this is a plausible
strategy in a specific framework for our choice of basis. Furthermore, I will present evidence that 
this should work \textit{better} for the kinds of cases we will face for medium
and heavy nuclei, that is, with unequal numbers of protons and neutrons.

The framework that we will work in is a proton-neutron coupling scheme, the details of 
which will be discussed in Chapter 5. The evidence will be the \textit{entropy of entanglement}
between protons and neutrons in atomic nuclei, which I will use to show that one can, in principle, efficiently
truncate the basis. The question of \textit{how} this can actually be done is still
under investigation. I will present, however, a code which carries out the 
coupling and give our results to date. While the project of efficiently calculating
nuclear properties using a truncated basis is not yet complete, I can
demonstrate significant progress towards it.

\section{Mathematical Background}
In this and the following section I give a brief introduction to the mathematical underpinning
of the nuclear many body problem. I assume an understanding of undergraduate quantum 
mechanics.

Nuclear physics is essentially a quantum 
many-body problem and much of nuclear physics can be explained by the 
(non-relativistic) N-body Schr\"{o}dinger equation:
\begin{equation}\label{tid}
	\hat{H}\Psi(\vec{r}_1,\vec{r}_2,\vec{r}_3,...,\vec{r}_N)=E\Psi(\vec{r}_1,\vec{r}_2,\vec{r}_3,...,\vec{r}_N).
\end{equation}	
Here $\hat{H}$ is the Hamiltonian, which includes kinetic energy 
and potential energy terms for each particle. The potential energy contains the 
nuclear interaction and/or other relevant 
interactions like the Coulomb force\cite{Johnson13}. $\Psi$ is our wave-function,  
a function of the position of every particle in the system. 
If we could solve the Schr\"{o}dinger equation for every system, the task of the 
theoretical nuclear physicist would be complete. However, in general, there is 
no analytic solution to (\ref{tid}) except for in a very small number of cases involving 
a great deal of symmetry and a small number of particles. In any practical situation 
where calculations of realistic nuclear structure are required, numerical methods are necessary. 

Towards this end, let's review the matrix formulation of quantum mechanics. 
Let us start with an eigenvalue equation of the form
\begin{equation}\label{hepe}
	\hat{H}\ket{\Psi}=E\ket{\Psi}.
\end{equation}
$\hat{H}$ is a Hermitian operator on a Hilbert space $\mathcal{H}$ and $\ket{\Psi}$ is 
a state vector which is a member of the 
Hilbert space. (A Hilbert space is a vector space with an inner product
that is complete in the norm)\cite{Hall}. By definition, any basis 
$\{\ket{\alpha}\}$ must satisfy the completeness relation
\begin{equation}
1 = \sum_\alpha \ket{\alpha}\bra{\alpha},
\end{equation}
thus any state vector $\ket{\Psi}$ can be expressed as the following, where 
$\Psi_\alpha \equiv \braket{\alpha}{\Psi}$
\begin{equation}
	\ket{\Psi} = \sum_\alpha \Psi_\alpha \ket{\alpha}.
\end{equation}
For this to be meaningful we will eventually need to choose a particular basis 
to work in. Starting again from (\ref{hepe}) and substituting this representation 
of our eigenstate,
\begin{equation}\begin{split}	
        \hat{H}\sum_\alpha \Psi_\alpha\ket{\alpha} = E\sum_\alpha \Psi_\alpha\ket{\alpha}
\end{split}\end{equation}
and projecting onto a dual basis state $\bra{\beta}$       
\begin{equation}\begin{split}
	\bra{\beta}\hat{H}\sum_\alpha \Psi_\alpha\ket{\alpha} &= \bra{\beta}E\sum_\alpha \Psi_\alpha \ket{\alpha} \\
        \sum_\alpha \bra{\beta}\hat{H}\ket{\alpha}\Psi_\alpha &= \sum_\alpha \braket{\beta}{\alpha} E\Psi_\alpha \\
        \sum_\alpha H_{\beta\alpha} \Psi_\alpha &= E\Psi_\beta,
\end{split}\end{equation}
where the matrix elements of $\hat{H}$ are defined to be
\begin{equation}
	H_{\beta\alpha} \equiv \bra{\beta}\hat{H}\ket{\alpha}.
\end{equation}

Now that we have represented the Schrodinger equation as a matrix eigenvalue problem, 
it becomes in principle straightforward to obtain important results such 
as the energy spectra and transition probabilities of nuclei: simply diagonalize
the Hamiltonian matrix to obtain the energy spectrum (eigenvalues) and 
wavefunctions (eigenvectors), which are used to compute observables such as
transition probabilities.

Complications include the fact there is really no way to write down a simple 
analytic Hamiltonian for the nuclear force as there is for classical forces 
such as electromagnetism. The nuclear force is fundamentally an exchange of mesons
requiring quantum field theory to properly describe\cite{eisenberg}. 
Furthermore, even once a  Hamiltonian is obtained, solving for its eigenvalues 
can be a non-trivial computational problem, since the  dimensions of the space 
can be in the many-billions. There are a number of popular computational 
methods for solving the many-body Hamiltonian, of which the configuration 
interaction method is an example. 

\section{Slater Determinants} 
A common practice in the configuration interaction model is to write the 
many-body basis states $\{\ket{\alpha}\}$ used above in terms of anti-symmetrized 
products of single-particle solutions to a single-particle Hamiltonian.
 Let's explore what this means.  

Because our wavefunctions $\Psi$ live in a vector space (more precisely, in a 
Hilbert space) 
we can always rewrite $\Psi$ in another basis of our choosing.
\begin{equation}
	\Psi(\vec{r}_1,\vec{r}_2,\vec{r}_3,...,\vec{r}_N)=\sum_\alpha c_\alpha
\Phi_\alpha(\vec{r}_1,\vec{r}_2,\vec{r}_3,...,\vec{r}_N)
\end{equation}
This is an N-body wavefunction with $N$ sets of coordinates, one for each particle.
The CI method is to choose our $\Phi$ to be a product of single-particle wave
functions,
\begin{equation}
    \{ \phi_a \},\ a=1,...,d,
\end{equation}
where $d$ is the number of single-particle states. These single-particle 
wavefunctions are chosen to be some convenient basis that will span the Hilbert 
space of interest.  I will discuss the exact nature of the single-particle states
in Section \ref{EISM}.
The normalization integral of a product wavefunction is simply the product of 
the integrals of each factor. However, since nuclei are composed of 
fermions (spin-1/2 particles), we must have antisymmetric wavefunctions. 
We therefore need the more complicated rule\cite{Dirac}:
\begin{equation}\label{sld}
	\Phi(\vec{r}_1,\vec{r}_2,\vec{r}_3,...,\vec{r}_N)=\frac{1}{\sqrt{N!}}
    \sum_{i_1,i_2,...,i_N}^N \epsilon_{i_1,i_2,...,i_N}\phi_1(\vec{r}_{i_1})
    \phi_2(\vec{r}_{i_2})\phi_3(\vec{r}_{i_3})...\phi_d(\vec{r}_{i_N}),
\end{equation}
\textit{viz.} the determinant of the matrix defined by $[\phi_a(i_b)]$ 
with a normalization factor, with index $a$ labeling the single-particle state 
and index $i_b$ labeling the particle number. The antisymmetry is encoded in the Levi-Civita symbol 
$\epsilon_{i_1,i_2,...,i_N}$\cite{Jeevanjee}. By defining our many-body wave-functions to be 
determinants of single-particle wavefunctions we are guaranteed antisymmetric 
many-body states. These specially constructed states are known as Slater determinants\cite{Wong}. 

\section{Second Quantization}
Occupation representation is an additional formalism that is used in our 
representation of wavefunctions. Here 
I will only state the relevant results of this so-called second quantization formalism. 
Many-body basis states
are no longer represented as the product of coordinate functions of single particle states.
Instead, a basis of single-particle states is chosen, perhaps the same set
$\{\phi_a\}$ as above, and the many-body states are represented in the abstract
as
\begin{equation}
    \Psi(\vec{r}_1,\vec{r}_2,\vec{r}_3,...,\vec{r}_N)\to\Psi{\{n_\phi\}} = \ket{n_1,n_2...n_d},
\end{equation}
where $n_a$ is the number of particles in the state $\phi_a$. Without any additional
structure, this representation lacks anti-symmetry; this is restored by representing
the state $\ket{n_1,n_2...n_d}$ as the action of a series of operators on the vacuum
state $\ket{0,0...0}$, or ``particle vacuum''. The operator which turns $\ket{0,0...0}$ into $\ket{1,0...0}$
is called the creation operator, and the operator which turns $\ket{1,0...0}$ into
$\ket{0,0...0}$ is called the annihilation operator. For fermions, these two generalized
ladder operators are defined implicitly through their anti-commutation relations\cite{Waleka}:
\begin{equation}\begin{split}
    \{\hat{a}_i,\hat{a}^\dagger_j\}&\equiv\hat{a}_i\hat{a}^\dagger_j+\hat{a}^\dagger_j\hat{a}_i
        =\delta_{ij}\\
    \{\hat{a}_j,\hat{a}_j\}&=\{\hat{a}^\dagger_i,\hat{a}^\dagger_j\}=0
\end{split}\end{equation}
These anti-commutation relations encode the properties of systems of fermions:
two fermions cannot occupy the same state, $(\hat{a}^\dagger_i)^2=0$, and 
therefore the maximum occupancy of a state $\phi_i$ is one. Thus, 
\begin{equation}\begin{split}
    \hat{a}^\dagger_1 \ket{1,0,...,0} &= 0, \\
    \hat{a}^\dagger_1 \ket{0,0,...,0} &= \ket{1,0,...,0},\\
    \hat{a}_1 \ket{1,0,...,0} &= \ket{0,0,...,0}, \\
    \hat{a}_1 \ket{0,0,...,0} &= 0,
\end{split}\end{equation}
and so on for any state $\ket{n_1,n_2...n_d}$. With this mathematically equivalent 
structure, we can replace our Slater determinants (\ref{sld}) with occupation 
representations using second quantization operators. Any many body state can
now be written
\begin{equation}\label{sld2}
    \Phi = \ket{n_1,n_2...n_d} = (\hat{a}^\dagger_1)^{n_1}...(\hat{a}^\dagger_d)^{n_d}\ket{0,...,0}.
\end{equation}
Since each state's occupation $n_i$ is
constrained to $0$ or $1$, many body states can be represented by bit strings. 
For example\cite{Johnson18},
\begin{equation}
    \hat{a}^\dagger_1\hat{a}^\dagger_4\hat{a}^\dagger_5\hat{a}^\dagger_7\ket{0} = \ket{1457} = \to
    1001101,
\end{equation}
where we are dealing with a four-body wavefunction in a basis with at least
seven single-particle states, and where I have used $\ket{0}$ here in place of $\ket{0,...,0}$
for brevity. This makes computer storage of wavefunctions trivial.
 
It can be shown that one-body operators can be expressed in the second quantization
formalism as\cite{Brussard}:
\begin{equation}
    \hat{O}=\sum_{ij}\bra{i}O\ket{j}\hat{a}^\dagger_i\hat{a}_j,
\end{equation}
where the sum is over all single-particle states. Similarly, two-body operators 
can be expressed\cite{Brussard}:
\begin{equation}
    \hat{O}=\frac{1}{4}\sum_{ijkl}\bra{ij}O\ket{kl}\hat{a}^\dagger_i\hat{a}^\dagger_j\hat{a}_k\hat{a}_l.
\end{equation}
The factor of one-fourth comes from the fact that the sum is over all single-particle 
states and would otherwise over count the number of (pairwise) interactions.
The Hamiltonian may be rewritten using these representations:
\begin{equation}\label{nuchamil}
	\hat{H}=\sum_{i,j}\hat{a}^\dagger_i\bra{i}T\ket{j}\hat{a}_j
    +\frac{1}{4}\sum_{i,j,k,l}\hat{a}^\dagger_i\hat{a}^\dagger_j\bra{ij}V\ket{kl}\hat{a}_l\hat{a}_k,
\end{equation}
where the first term represents the one-body interactions such as kinetic energy or the Coulomb 
interaction with an external field, and the second term represents two-body interactions 
(forces between interacting particles) and contains the nuclear force. This formalism is key to 
the way this model is translated into powerful computer codes which can be used for a number 
of different calculations. 

\section{Effective Interactions and Shell Models}
\label{EISM}
A truly realistic model of the nucleus would necessarily involve the description of the exchange of 
mesons\cite{eisenberg}. The basis of single particle states $\{\phi_a\}$ that 
are used in shell model methods are instead taken to be time independent
solutions to quantum potentials which attempt to reproduce the results of a
full field theory.
Some variation of a mean-field approximation is used to motivate the choice
of single particle states.
For example, this could be a harmonic oscillator, or the more sophisticated
Woods-Saxon potential\cite{Brussard}. In both cases, and in all cases considered
here, the defining single-particle potential is chosen to exhibit rotational invariance,
resulting in states of good angular momentum. In fact we will always choose
single particle states with good harmonic oscillator 
quantum numbers: the radial quantum number $n$, the orbital angular momentum $l$,
the total angular momentum $j$, the
z-component of the angular momentum $j_z\equiv m$ and the parity quantum number
$\pi$. Parity $\pi=(-1)^l$ and $m=-j,-j+1,...,j-1,j$. 

It is common practice to use atomic spectroscopic notation to label the nuclear
single-particle orbits (states). Here we will take the convention $nl_j$ where
$l$ is replaced with the letters
\begin{equation}\begin{split}
       l &= 0,1,2,3,4,5,...\\
         &= s,p,d,f,g,...
\end{split}\end{equation}
For example the first few low-lying levels of a harmonic oscillator-like single-particle
potential are shown in Table \ref{spo}. (Degeneracy indicates the numbers of states with unique $m$ in each orbit $nl_j$.
    Max fill indicates the sum of all degeneracies up to that orbit and is the maximum
    number of nucleons of each species that can fill the space up to and including that orbit.). There is a distinction between an orbit $nl_j$ and a state which also includes the 
z-component of the angular momentum $m$. An orbit labeled as $nl_j$ has $2j+1$ possible
states since $m$ is restricted to $-j, -j+1, ..., j-1, j$. States within the same orbit are degenerate 
in energy (at least in the absence of any external electromagnetic fields), with $2j+1$ being the
degeneracy.
\begin{table}
    %\centering
    \caption{Harmonic Oscillator States with Spin Orbit Coupling 
    }
    \label{spo}


\begin{tabular}
    {c c c c c c c c}
    \hline 
    \hline
Orbit     & n & l & j   & m         & Parity & Degeneracy & Max fill  \\
    \hline
$0s_{1/2}$& 1 & 0 & 1/2 & 1/2, -1/2 & + & 2 & 2* \\
    \hline
$1p_{3/2}$& 1 & 1 & 3/2 & 3/2,...,-3/2 & - & 4 & 6 \\
$1p_{1/2}$& 1 & 1 & 1/2 & 1/2, -1/2    & - & 2 & 8* \\
    \hline
$1d_{5/2}$& 1 & 2 & 5/2 & 5/2,...,-5/2 & + & 6  & 14   \\
$2s_{1/2}$& 2 & 0 & 1/2 & 1/2, -1/2    & + & 2  & 16  \\
$1d_{3/2}$& 1 & 2 & 3/2 & 3/2,...,-3/2 & + & 4  & 20*  \\
    \hline
$1f_{7/2}$& 1 & 3 & 7/2 & 7/2,...,-7/2 & - & 8 & 28* \\
    \hline
$2p_{3/2}$& 2 & 1 & 3/2 & 3/2,...,-3/2 & - & 4 & 32 \\
$1f_{5/2}$& 1 & 3 & 5/2 & 5/2,...,-5/2 & - & 6 & 38 \\
$2p_{1/2}$& 2 & 1 & 1/2 & 1/2,    -1/2 & - & 2 & 40 \\
$1g_{9/2}$& 1 & 4 & 9/2 & 9/2,...,-9/2 & + & 10& 50* \\
    \hline
    \hline

\end{tabular} 

\end{table} 

In the harmonic oscillator basis, there are an infinite number of single-particle
states. We cannot keep track of an infinite number of states; this would require 
bit string representations of our many-particle states that are infinitely long. 
We therefore consider only a finite number of single-particle states, and leave out
the rest. We leave out single-particle states which have a very 
low probability of being occupied, as well as those with a very high occupation
probability. This second statement may seem counter-intuitive. How can we ignore 
states which are occupied? Because these states are assumed to always be occupied,
we can account for their contribution to the single-particle energies
and two-body interactions while leaving the states out of the active model space. 
To determine which low-lying states are likely to always be filled, we rely on
``magic numbers''.
An important observational fact is that the binding energy of certain nuclei with certain 
numbers of nucleons tend to be especially stable.
Nuclei with 2, 8, 20, 28, 50, 82, or 126 protons or neutrons tend to have 
a higher binding energy than nuclei with one proton or one neutron above these
so called magic numbers \cite{Segre}. Magic numbers in Table \ref{spo} are marked
with an asterisk. 

In shell model codes, the vacuum state is redefined to be an inert core of nucleons, 
and the model space to be a finite space of particle-hole excitations near a 
Fermi surface\cite{Heyde}, usually determined by one of the magic numbers.
Particles in the set of active, interacting nuclei occupy the ``valence'' space. 
For example, we can imagine an inert core of 8 noninteracting protons and 8 noninteracting
neutrons, which can be thought of as a frozen $^{16}$O nucleus. Then, orbiting above
this core we can imagine a fixed number of interacting nucleons, valence protons and neutrons.
If we restrict the number of valence protons and neutrons to between 0 and 12, then
we can represent the entire valence space with just the $d_{3/2}$, $d_{5/2}$ and 
$s_{1/2}$ orbitals. This is called the \textit{sd}-shell model space because the s($l=0$) and d($l=2$) 
orbital quantum numbers are all that appear in the truncated model space. All
quantum numbers for this single-particle model space are shown in Table \ref{table: sd.spo}.
(These
    orbits lie directly above an inert core of 8 protons and 8 neutrons and can
    support up to 12 valence protons and 12 valence neutrons. Each orbit has $2j+1$ 
    possible values of $m$, this is the degeneracy.
    The rightmost column indicates the maximum number of particles that could fill
    the model space up to that orbit.)
The overall effect of the inert $^{16}$O core is to add an effective energy to 
Hamiltonian.\cite{Heyde}.   

\begin{table}
    \caption{\textit{sd}-Shell Single Particle Orbits}
    \label{table: sd.spo}
%\centering
\begin{tabular}
    {c c c c c c c}
    \hline 
    \hline
Orbit & l & j & m & Degeneracy & Max fill  \\
    \hline
$d_{3/2}$      & 2 & 3/2  & 3/2, 1/2, -1/2, -3/2            & 4  & 4   \\
$d_{5/2}$      & 2 & 5/2  & 5/2, 3/2, 1/2, -1/2, -3/2, -5/2 & 6  & 10  \\
$s_{1/2}$      & 0 & 1/2  & 1/2, -1/2                       & 2  & 12  \\
    \hline
    \hline
\end{tabular}

\end{table} 

Another common space that we will be using is the $f_{7/2}$, $p_{3/2}$, $f_{5/2}$, and $p_{1/2}$ model space, 
often referred to as the pf-shell model space. The single-particle orbits 
for this space are given in Table \ref{table: fp.spo}. (Same conventions as in Table
\ref{table: sd.spo}. This model space has an inert core of 20 protons and 20 neutrons
and has 20 unique proton and 20 unique neutron states.)

\begin{table}
    \caption{\textit{pf}-Shell Single Particle Orbits }
    \label{table: fp.spo}
%\centering
\begin{tabular}
    {c c c c c c c}
    \hline 
    \hline
Orbit & n & l & j & m & Degeneracy & Max fill  \\
    \hline
$1f_{7/2}$    & 1  & 3 & 7/2  & 7/2, ..., -7/2      & 8  & 8   \\
$2p_{3/2}$    & 2  & 1 & 3/2  & 3/2, 1/2, -1/2, -3/2& 4  & 12  \\
$1f_{5/2}$    & 1  & 3 & 5/2  & 5/2, ..., -5/2      & 6  & 18  \\
$2p_{1/2}$    & 2  & 1 & 1/2  & 1/2, -1/2           & 2  & 20    \\
    \hline
    \hline
\end{tabular}
\end{table} 

To account for the state left out of the model space,  
an effective two-body interaction can be computed. By enforcing that the 
effective interaction acting on the model wavefunction reproduces the same
results as the full interaction on the full wavefunction, one can 
expand the effective interaction into a series of the form\cite{Heyde}
\begin{equation}\label{eff}
    V^{eff} = V + V\frac{\hat{Q}}{E - H_0}V + ...
\end{equation}
where $\hat{Q}$ is a projection operator onto the part of the full Hilbert space
that was left out of the model. 

In our work we rely on commonly used effective two-body interactions computed by others,
such as the Universal \textit{sd}-shell interaction version B (USDB)\cite{usdb}. This interaction is not derived using an expansion 
like (\ref{eff}), however. The USDB interaction and others like it are phenomenological
interactions, meaning that it may have started from an effective calculation like
equation (\ref{eff}), but then matrix elements were tuned in order to give better
fits to widely available measured masses and resonance energies (e.g. from the National Nuclear Data Center (NNDC))\cite{nndc}. The USDB interaction 
and other widely used interaction matrix elements are lists of quantities of the form
\begin{equation}
    \bra{i,j}V\ket{k,l},
\end{equation}
as appears in equation (\ref{nuchamil}). 

\section{Bringing it All Together: the Interacting Shell Model}

To construct a many-body basis in the interacting shell model, we begin with 
a shell model space. I will use the \textit{sd}-shell model space to illustrate
the development of this section. This model space has an inert core of
8 protons and 8 neutrons. The contribution from these inactive particles is accounted
for in the single-particle energies and two-body matrix elements of the Hamiltonian. 
We have three single particle orbits ($d_{3/2}$, $d_{5/2}$, and $s_{1/2}$), 
each with a $2j+1$ degeneracy. 
We often assume that protons and neutrons have the same set of single-particle orbits.
Each orbit has a single-particle energy associated with it. Given the degeneracy of each orbit, 
we have $12$ possible single-particle states, as shown in Table \ref{sdstates}.
These single-particle states span the valence space, that is, the space of 
all possible single-particle wavefunctions within the \textit{sd}-shell model space.
Many-particle wavefunctions, i.e. proton or neutron Slater determinants, can be represented with a 
bit string twelve bits long. If there are $N$ fermions of a given species in the model space,
then there are $_{12}C_N$ (twelve choose N) possible Slater determinants. For 
example, a state with three protons (or neutrons) could be
\begin{equation}\label{slaterex}
    \hat{a}_1^\dagger\hat{a}_2^\dagger\hat{a}_3^\dagger\ket{0} = \ket{123} = \ket{111000000000} \to 111000000000,
\end{equation}
but there are $_{12}C_3=220$ possible 3-nucleon Slater determinants, many of 
them degenerate in energy.
The single-particle energy of such a state (i.e. ignoring nucleon-nucleon interactions) would be
the sum of the single-particle energies of each nucleon.

\begin{table}[h]

    \caption{\textit{sd}-Shell Single-particle States}
    \label{sdstates}
%\centering
\begin{tabular}
    {c c c c c c c c}
    \hline 
    \hline
State \# & Orbit & l & $\pi$ (Parity) & j & m \\
1       & $d_{3/2}$ & 2 & + & 3/2 & +3/2 \\
2       & $d_{3/2}$ & 2 & + & 3/2 & +1/2 \\
3       & $d_{3/2}$ & 2 & + & 3/2 & -1/2 \\
4       & $d_{3/2}$ & 2 & + & 3/2 & -3/2 \\
5       & $d_{5/2}$ & 2 & + & 5/2 & +5/2 \\
6       & $d_{5/2}$ & 2 & + & 5/2 & +3/2 \\
7       & $d_{5/2}$ & 2 & + & 5/2 & +1/2 \\
8       & $d_{5/2}$ & 2 & + & 5/2 & -1/2 \\
9       & $d_{5/2}$ & 2 & + & 5/2 & -3/2 \\
10      & $d_{5/2}$ & 2 & + & 5/2 & -5/2 \\
11      & $s_{1/2}$ & 0 & + & 1/2 & +1/2 \\
12      & $s_{1/2}$ & 0 & + & 1/2 & -1/2 \\
    \hline
    \hline
\end{tabular}

\end{table} 

Equation (\ref{slaterex}) is a many-proton or many-neutron Slater determinant. The
wavefunction of an atomic nucleus, containing both protons and neutrons, is 
constructed by coupling together proton and neutron Slater determinants:
\begin{equation}\label{abasis}
    \ket{\alpha} = \ket{j_p,m_p}\otimes\ket{j_nm_n},
\end{equation}
where $\ket{j_p,m_p}=(\hat{a}^\dagger_1)^{n_1}...(\hat{a}^\dagger_d)^{n_d}\ket{0,...,0}$ 
is a particular proton Slater determinant such as (\ref{slaterex}), and similarly 
for the many-neutron state $\ket{j_n,m_n}$. 

The number of nuclear Slater determinants can quickly become very large.
The \textit{sd}-shell model space, with 12 single-particle states, has a maximum
of $_{12}C_6=924$ single-species Slater determinants, for a maximum of $924^2=853776$
proton-neutron Slater determinants. Fortunately, symmetries of the Hamiltonian,
specifically selection rules for quantum numbers allow us to ignore certain 
combinations of these states, allowing us to reduce the overall size of the basis.
If the Hamiltonian is rotationally invariant, then
both the total angular momentum operator $\hat{J}^2$ and the z-component of angular
momentum operator $\hat{J}_z$ commute with the Hamiltonian. This makes choices 
of basis where either $J$ or $J_z=M$ is held constant for all basis states convenient\cite{Johnson13},
since for any coupled basis states $\ket{\alpha,M_j}$ with good total $M$
\begin{equation}
    \bra{\beta,M_\beta}\hat{H}\ket{\alpha,M_\alpha} = 0,
\end{equation}
whenever $M_\alpha\neq M_\beta$. This holds true for states of fixed $J$ as well, but in general
this is less convenient because whereas $M$ is an additive quantum number,
$J$ is not and requires more cumbersome combinations of Slater determinants\cite{Johnson13}.
Therefore for (\ref{abasis}) we often choose the so-called M-scheme basis where
basis states are coupled up to good $M$:
\begin{equation}
    \ket{\alpha} = [\ket{j_pm_p}\otimes\ket{j_nm_n}]_M \equiv \ket{j_pj_n;M},
\end{equation}
where $M=m_p+m_n$. In terms of our bit representations this means selecting to 
combine proton slater determinants with neutron Slater determinants that will
result in fixed total $M$. For example, an atomic nucleus in the \textit{sd}-shell
model space with a single valence proton and a single valence neutron (i.e. $^{18}$F)
for $M=0$ can only have coupled Slater determinants such as 
\begin{equation}\begin{split}
    \ket{1}\otimes\ket{4}=\ket{j_p=3/2,m_p=+3/2}&\otimes\ket{j_n=3/2,m_n=-3/2},\\
    \ket{4}\otimes\ket{1}=\ket{j_p=3/2,m_p=-3/2}&\otimes\ket{j_n=3/2,m_n=+3/2},\\
    \ket{2}\otimes\ket{3}=\ket{j_p=3/2,m_p=+1/2}&\otimes\ket{j_n=3/2,m_n=-1/2},\\
    \ket{3}\otimes\ket{2}=\ket{j_p=3/2,m_p=-1/2}&\otimes\ket{j_n=3/2,m_n=+1/2},\\
\end{split}\end{equation}
etc., where $\ket{i}$ is a single particle occupying the $i^{th}$ single-particle 
state in Table \ref{sdstates}. In total there are $28$ possible $M=0$ coupled
Slater determinants for $^{18}$F in the \textit{sd}-shell model space, translating
to a basis and Hamiltonian matrix with a dimension of $28$. 

The M-scheme is highly advantageous. However, if we wish to truncate the basis,
the J-scheme becomes preferable. I will explain this in detail in a later chapter, 
after justifying the need for a truncation of the basis.





 

