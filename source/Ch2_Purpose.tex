\chapter{STATEMENT OF THE PROBLEM}
\label{chap:purpose}

In order to solve the nuclear many-body problem, the Schrodinger equation is recast
as a matrix eigenvalue problem.
We use an M-scheme configuration interaction code called BIGSTICK \cite{Johnson13}\cite{Johnson18}. 
Despite the fact that the interaction files used to build the basis for the space are
semi-empirical, BIGSTICK and codes like it are still numerically exact, that is, 
they don't use any numerical approximations once the Hamiltonian is built, and 
the basis is not truncated. This means that the error relative to experiment is
due to the interaction the code uses and not the method for solving the problem. 
The error in the results are fairly low, around a few 
hundred keV, while low-lying states tend to be a few MeV in magnitude.
BIGSTICK and configuration interaction codes like it can still be extremely computationally 
expensive to run for large nuclei, even after an order of magnitude improvement
using the M-scheme. Table \ref{memory} contains a short list of the storage space 
(RAM) required to store just the nonzero matrix elements
for various nuclei in different phenomenological model spaces.  
A discussion of the dimensions of even heavier nuclei can be found in Chapter 7.

\begin{table}[h!]
    \caption{Sample Shell Model Dimensions}
    \label{memory}
%\centering
\begin{tabular}
    {c|c|c|c}
    \hline
    \hline
    Nuclide   & Space & Basis dim. & Storage\cite{Johnson13,Johnson18} (GB)\\
    \hline 
    $^{28}$Si & $sd$ & $9.4\times 10^4$ & 0.2 \\
    $^{52}$Fe & $pf$ & $1.1 \times 10^8$ & 720 \\
    $^{56}$Ni & $pf$ & $1.1 \times 10^9$ & 9600 \\
    \hline
    \hline
\end{tabular}

\end{table} 

Because of the intense computational cost of solving nuclear
structure problems, and the limited computer resources available,
it is necessary to approximate solutions via truncated model spaces.
The purpose of this thesis is to investigate the viability of 
truncating the nuclear many-body basis in a particular framework, 
a basis of coupled proton and neutron Slater determinants, and 
to truncate the basis by selecting the most important states. 
The best method for doing so has yet to be determined, but later in Chapter 5 I will
explain the attempt we have made. First I will provide
evidence that such a method could be viable. 

To do so I analyze the distribution of wavefunction coefficients. This 
investigation was inspired by a similar study of
light to medium mass nuclei using a singular value decomposition (SVD)
of the wavefunction coefficients\cite{Papenbrock03,Papenbrock04,Papenbrock05}.
In those studies, shell model basis states were approximated
by truncating the basis in a coupled proton-neutron scheme. 
Ground state energies in the \textit{sd} and \textit{pf} shell, as well as 
low-lying states, were shown to converge exponentially with basis dimension to those 
computed with the untruncated basis.
The authors examined mostly
$N=Z$ nuclei. The authors also provided no explanation for the 
cause of the exponential convergence, and restricted themselves to relatively 
low dimensional problems.
In the next chapter, I will take a similar approach by examining the 
distribution of wavefunction coefficients through the proton-neutron entanglement
entropy.  


\section{Theoretical Motivation}
We can always expand a wavefunction into any basis. Suppose we have 
some initial representation of our wavefunction,
\begin{equation} \label{eqn: decomp1}
    \ket{\Psi}=\sum_{i_p,j_p} \Psi_{i_p,j_n} \ket{i_p}\ket{j_n},
\end{equation}    
where the uncoupled basis states are Slater determinants:
\begin{equation}\begin{split}
    \ket{i_p} \textrm{ and }\ket{j_n}, 
\end{split}\end{equation}
where $i_p = 1, d_p$ and $j_p = 1, d_n$. Here, $d_p$ and $d_n$ are the number 
of proton and neutron basis states, respectively.

We choose to decompose our wavefunction $\ket{\Psi}$ into pure proton 
$\ket{\pi_a}$ and pure neutron $\ket{\nu_b}$ wavefunctions which are 
related to the old basis by some 
unitary transformation, 
\begin{equation}\label{unitary}
    \ket{\pi_a}=\sum_{i_p}U_{ai_p}^\pi \ket{i_p},
\end{equation}
and similarly for $\ket{j_n}$ and $\ket{\nu_b}$.
Thus we can find
\begin{equation}\label{eqn: decomp2} \begin{split}
    \ket{\Psi}  =\sum_{a,b} \tilde{\Psi}_{ab}\ket{\pi_a}\ket{\nu_b},
\end{split}\end{equation}  
where $a$ enumerates proton single-particle states and $b$ enumerates 
neutron single-particle states. If the sum is taken over all states, (\ref{eqn: decomp2}) is equivalent to 
(\ref{eqn: decomp1}). The goal, however, is to find a basis in which the 
sum (\ref{eqn: decomp2}) can be truncated, thus reducing the size of the problem.

\begin{equation}\label{eqn: decomptrunc}
\ket{\Psi} = \sum_{a,b = 1}^{N} \tilde{\Psi}_{ab}\ket{\pi_a}\ket{\nu_b},\ \ N << \min(d_p,d_n).
\end{equation}
In order for the truncated sum to be useful, the terms we exclude must 
be small compared to the overall wavefunction. 
If we have some normalized representation of a wavefunction, 
\begin{equation}\label{refEE1}
    \ket{\Psi} = \sum_\alpha c_\alpha \ket{\Phi_\alpha},
\end{equation}
then the sum of the weights $|c_\alpha|^2$ is unity:
\begin{equation}
    1 = \sum_\alpha|c_\alpha|^2,
\end{equation}
and the distribution of the weights $|c_i|^2$ tells us how much the sum (\ref{refEE1})
can be truncated in the basis ${\ket{\Phi_i}}$. 

In a basis of coupled single particle states as in (\ref{eqn: decomp1}), 
the coefficients form a non-symmetric matrix $\Psi_{i_p,j_n}$. However, it was shown in the
studies cited above\cite{Papenbrock03,Papenbrock04,Papenbrock05} that a singular value
decomposition\cite{NumRec} (SVD) of this matrix would transform $\Psi_{i_p,i_n}$ into a 
diagonal matrix such that
\begin{equation}\label{refEE2}
    \ket{\Psi} = \sum_i \gamma_i \ket{\tilde{i_p}}\ket{\tilde{i_n}},
\end{equation}
where $\gamma_i=\tilde{\Psi}_{ii}$ are the diagonal elements of the transformed matrix of $\Psi_{i_p,j_n}$,
and $\ket{\tilde{i_p}}$ and $\ket{\tilde{i_n}}$ are some undetermined basis states.
(See Appendix \ref{appendix: svd} or Chapter 3 for a more detailed explanation of 
singular value decompositions.)
If such an expansion could be found, then, as for the simple case in (\ref{refEE1}),
the distribution of expansion coefficients $\gamma_i$ would tell us how much 
the expansion (\ref{refEE2}) could be truncated.
The problem of course is determining the best basis, an as-yet unsolved problem.
Nonetheless, there is still something to be learned from singular value decompositions.
Just as eigenvalues are invariant under a change of basis, the ``eigenvalues''
$\gamma_i$ are also invariant under a change of basis. (In fact these SVD eigenvalues
and the SVD itself are generalizations of eigenvalue decompositions.) Therefore,
we can compute the SVD eigenvalues of a known expansion, such as from the nuclear
wavefunctions from an M-scheme method (e.g. BIGSTICK), and at the very least
we can determine whether or not an truncation is possible.

\section{Method}
I conduct two related studies: In Chapter 3, I will examine the distribution of
nuclear wavefunction amplitudes, from which we extract the
proton-neutron entanglement entropy. I provide evidence that the creation of approximate wavefunctions
through coupling of proton and neutron wavefunctions may be a 
valuable method, and I test a postulate which predicts
that this proton-neutron coupling scheme will be even more effective for
nuclei where the number of neutrons is much larger than the number of 
protons, i.e. for large isospin. Chapter 3 also attempts to investigate the
cause of this behavior via the entanglement entropy.
Then in Chapter 4, I will decompose BIGSTICK
wavefunctions into their pure-proton and pure-neutron components by projecting
the existing wavefunctions onto the appropriate operators. This, I will argue, 
demonstrates directly that a truncated basis of coupled proton and neutron 
Slater determinants can accurately reproduce nuclear wavefunctions.




