\chapter{PROTON-NEUTRON DECOMPOSITION}
\label{chap:decomp}

In the previous chapter, we studied the distribution of wavefunction coefficients
in order to provide evidence that there exists a basis in which our nuclear
wavefunctions could be truncated, thus allowing more efficient representation
of the wavefunctions. Ultimately our goal is to find a basis in which our 
Hamiltonian can be truncated.
In this chapter I will examine a decomposition of wavefunctions which
corresponds to a particular framework for computing truncated Hamiltonian matrix
elements.

We will be assuming that our Hamiltonian can be decomposed as
\begin{equation}\label{htotal}
	\hat{H} = \hat{H}_{proton} + \hat{H}_{neutron} + \hat{H}_{proton-neutron}
\end{equation}
where $\hat{H}_{proton}$ contains all interactions between protons, 
$\hat{H}_{neutron}$ contains all interactions between neutrons, and $\hat{H}_{proton-neutron}$ contains
all remaining interactions between protons and neutrons. Then we call the 
eigenstates of the pure-proton Hamiltonian $\ket{\pi}$, as in
\begin{equation}\label{hppp}
	H_{proton}\ket{\pi} = E_p\ket{\pi},
\end{equation}
and the eigenstates of the pure-neutron Hamiltonian are equivalently
define for $\hat{H}_{neutron}$ and $\ket{\nu}$. 
Roughly speaking, our proposed truncation 
scheme is to compute the full Hamiltonian (\ref{htotal}) in a basis
built up from the coupled eigenstates of $\hat{H}_{proton}$ and $\hat{H}_{neutron}$.
It was therefore worth decomposing existing wavefunctions $\ket{\Psi}$
into the eigenstates of both pure-proton interaction operators and 
pure-neutron interaction operators. This will tell us how whether or not
truncating a basis of coupled pure-proton and pure-neutron wavefunctions
will be viable. To see why, consider an arbitrary wavefunction $\ket{\Psi}$.
To expand $\ket{\Psi}$ into a basis $\{\ket{\alpha}\}$, one first computes the
coefficients
\begin{equation}
    c_\alpha = \braket{\alpha}{\Psi}.
\end{equation}
If some significant subset of the coefficients $\{\ket{\alpha}\}$ are much smaller than the rest,
then the full wavefunction,
\begin{equation}
    \ket{\Psi} = \sum_\alpha c_\alpha \ket{\alpha},
\end{equation}
could be represented by just a few terms. What we care about is the magnitude of
the coefficients $\{\ket{\alpha}\}$, thus we could instead compute
\begin{equation}\label{coef2}
    |c_\alpha|^2 = \braket{\Psi}{\alpha}\braket{\alpha}{\Psi},
\end{equation}
where the object $\ket{\alpha}\bra{\alpha}$ is a projection operator.

We computed the equivalent set of coefficients (\ref{coef2}) for our
nuclear wavefunctions projected onto the eigenstates of $\hat{H}_{proton}$ and
$\hat{H}_{neutron}$.
This was done using existing
capabilities in BIGSTICK to compute transition strength functions. In particular I
used the option to decompose wavefunctions onto scalar operators\cite{Johnson18}.
To do this, we first solve for the wavefunction $\ket{\Psi}$ for some nucleus.
Then BIGSTICK solves equation (\ref{hppp}) and the equivalent equation for the
neutron Hamiltonian, using its Lanczos capabilities to find low-lying 
eigenstates $\ket{\pi_a}$ (or $\ket{\nu_a}$). Then, the nuclear wavefunction is projected
onto these eigenstates in the following manner:
\begin{equation}\begin{split}
	\braket{\pi_a}{\Psi} &= \bra{\pi_a}\sum_{a,b}\tilde{\Psi}_{ab}\ket{\pi_a}\ket{\nu_b} \\
		              &= \sum_{b}\tilde{\Psi}_{ab}\ket{\nu_b}.
\end{split}\end{equation}
The strengths of these projections is then computed as
\begin{equation}\begin{split}
        \braket{\Psi}{\pi_a'}\braket{\pi_a}{\Psi} &= \sum_b \tilde{\Psi}^*_{a'b}\tilde{\Psi}_{ab} \\
 		&= \sum_b^{N_{keep}} |\Psi_{ab}|^2\\
        & \equiv c_a^2.
\end{split}\end{equation}
The quantities $c_a^2$ are the strengths of the projection of a wavefunction $\ket{\Psi}$
onto the proton-proton interaction eigenstate $\ket{\pi_a}$. Since we are assuming a set of normalized 
wavefunctions in our basis, 
\begin{equation}
1 = \sum_a \braket{\pi_a}{\pi_a},
\end{equation}
and thus the strengths $c_a^2$ are the fractions of the total wavefuntion $\ket{\Psi}$
contained in the state $\ket{\pi_a}$.

I projected nuclear wavefunctions onto the eigenstates of the proton Hamiltonian
of the neutron Hamiltonian and plotted the strengths $c_X^2$ for each interaction
as a function of the eigenstate number. A typical set of results in shown in
Figure \ref{fig: pndecomp1} and Figure \ref{fig: pndecomp2}.
The vertical axis is the strength of the decomposition onto the eigenstate
of either the proton Hamiltonian or the
neutron Hamiltonian with the $X^{th}$ lowest energy. Each plots has two sets
of values, one for each of two particle-hole conjugate nuclei in the \textit{sd}-shell:
one being the proton (PP) decomposition and the other being the neutron (NN) decomposition.

There are two important conclusions to be drawn from these decompositions. The first is that the strengths
fall off rapidly in the first handful of states. This suggests than
a truncation of the coupled pure-proton and pure-neutron eigenstates could a viable 
method for obtaining approximate wavefunctions. However it is already known that 
this is the case for light nuclei\cite{Papenbrock03,Papenbrock04,Papenbrock05}. The second
important feature is apparent when plotting the strength function
decompositions for particle-hole conjugate nuclei on the same plot. We find that
even though these pairs of nuclei have the same dimensions in their shell model spaces,
nuclei for which $N>Z$ have strengths which are significantly lower than their 
$N=Z$ conjugates. This is further evidence that the same proposed truncation scheme
may be even more effective in heavier nuclei where $N>Z$.

\begin{figure}
    \centering
    \includegraphics[width=4.5in]{Figures/decomp_na}
    \caption{Strength decomposition of nuclear wavefunctions into eigenstates of
    the proton-proton or neutron-neutron interaction for two particle-hole conjugate
    nuclei in the \textit{sd}-shell. Strengths fall off exponentially. } 
    \label{fig: pndecomp1}
\end{figure}

\begin{figure}
    \centering
    \includegraphics[width=4.5in]{Figures/decomp_na_ln}
    \caption{Strength decomposition of nuclear wavefunctions into eigenstates of
    the proton-proton or neutron-neutron interaction for two particle-hole conjugate
    nuclei in the \textit{sd}-shell. The decomposition strengths of $^{28}$Na, with an unequal number of protons
    and neutrons, fall off faster than its conjugate, suggesting that fewer states
    are necessary to represent it. } 
    \label{fig: pndecomp2}
\end{figure}
