\appendices
%\appendix
%
% If you only have one appendix, you should change the above to:
%\appendix
%
\chapter{Supplementary Notes}
\label{appendix: vc}
\section{Vector Coupling Coefficients}

Vector coupling coefficients, commonly referred to as Clebsch-Gordan coefficients
\cite{Shankar} are simply the coefficients of an expansion of a coupled angular 
momentum state with fixed total $J$ in a basis of coupled states 
$\ket{j_1m_1,j_2m_2}\equiv\ket{j_1m_1}\otimes\ket{j_2m_2}$. That is to say,
in the expansion
\begin{equation}
\ket{j_1j_2;JM}=\sum_{m_1m_2}\ket{j_1m_1,j_2m_2}\braket{j_1m_1,j_2m_2}{j_1j_2;JM},
\end{equation}
the inner products $\braket{j_1m_1,j_2m_2}{j_1j_2;JM}$ are the vector coupling
coefficients. This are often written as
\begin{equation}
    (j_1m_1j_2m_2|JM)\equiv \braket{j_1m_1,j_2m_2}{j_1j_2;JM}
\end{equation}
so that
\begin{equation}
    \ket{j_1j_2;JM} = \sum_{m_1m_2} (j_1m_1j_2m_2|JM) \ket{j_1m_1,j_2m_2}.
\end{equation}

\section{3-j Symbol}
An alternative to the vector-coupling coefficients are the so called three-J symbols
(3-j symbols). They are defined in terms of the vector coupling coefficients for the
coupling of three angular momenta\cite{Edmonds}:
\begin{equation}
    \begin{Bmatrix}j_1&j_2&j_3\\m_1&m_2&m_3\end{Bmatrix} \equiv
        (-1)^{j_1-j_2-m_2}(2j_2+1)^{-1/2}(j_1m_1j_2m_2|j_1j_2j_3-m-3).
\end{equation}

\section{6-j Symbols and 9-j Symbols}
Six-J and Nine-9 symbols are generalizations of vector coupling
coefficients which often appear in calculations involving complex angular
momentum algebra. I refer the reader to Edmonds\cite{Edmonds} for a complete description.

\section{Singular Value Decomposition}
\label{appendix: svd}
A singular value decomposition is a eigenvalue decomposition, 
generalized to non-square non-symmetric matrices. 
The method of singular value decomposition comes from a theorem 
which tells us that any $M \times N$ matrix A with $M \geq N$ can be written as
 the product of an $M \times N$ column-orthogonal matrix U, an $N \times N$ diagonal 
 matrix W with positive or zero elements (the singular values), and 
 the transpose of an $N \times N$ orthogonal matrix V\cite{NumRec}.
That is,
\begin{equation}
	A_{M \times N} = U_{M \times N} W_{N \times N} V_{N \times N}^T
\end{equation}
or in matrix notation,
\begin{equation}
	A_{ij} =  \sum_{a=1}^N U_{ia}  \sum_{b=1}^{N} W_{ab} V_{jb}
\end{equation}
Which, for the case where $M=N$, simply states that we can write a matrix $A$ as
a unitary transformation of a diagonal matrix containing the eigenvalues of $A$:
\begin{equation}
	A_{ij} = \sum_{ab}U_{ia}W_{ab}U_{jb}.
\end{equation}


\section{Lanczos Method}

This is an overview of the Lanczos method\cite{lanczos} used in this research. The 
Lanczos Algorithm is used to obtain the extremal eigenvalues of an $n \times n$
Hermitian matrix in $m<n$ linear operations. This is done by transforming the
Hermitian matrix into a truncated tridiagonal matrix containing the extremal 
eigenvalues,

\begin{equation}
    T = V^\dagger H V
\end{equation}

We begin with an $n \times n$ real-symmetric matrix $H$ and a fixed number $k<<n$
of iterations to carry out. Generate a random normalized vector $v(0)$ of dimension 
$n$ and carry out the following procedure:

\begin{verbatim}
    FOR all Lanczos iterations
        |w> = H|v(i)>
        a(i) = <v(i)|w>

        ! Orthogonalize against initial vector
        |w> = |w> - a(i)|v(i)>  

        IF (i>1), FOR all previous iterations
            ! Orthogonalize against prior vectors
            |w> = |w> - <v(j)|w>|v(j)> 

        b(i) = |w|
        |v(i+1)> = |w>/b(i)            
\end{verbatim}    

After $k$ iterations, the vectors $v(i)$ form the transformation matrix $V$ and
$a$ and $b$ contain the diagonal and subdiagonal elements of the tridiagonal 
matrix $T$.  

This algorithm can be used to extract the extremal eigenvalue and eigenvectors 
of $H$. By diagonalizing $T$, one can find $Tx = \lambda x$ where $x$ and $\lambda$
are the eigenvectors and eigenvalues of $T$. $H$ and $T$ are \textit{similar} matrices
and therefore have the same eigenvalues. Simultaneously $H\psi = \lambda \psi$ and 
$Tx = \lambda x$, so
\begin{equation}\begin{split}
    V^\dagger H Vx &= \lambda x \\
    VV^\dagger H Vx &= V\lambda x \\
    H Vx &= \lambda Vx, 
\end{split}\end{equation}
thus $\psi = Vx$ tells us how to obtain the eigenvectors of $H$ given the 
eigenvectors of $T$.

\section{Variational Principle}
\label{VP}
The Variational Principle is an extremely useful theorem in quantum mechanics as it
is the basis of many methods. 
The principle states that the expectation value of the Hamiltonian in any state is
greater than or equal to the expectation value of the Hamiltonian in the ground
state. Suppose we have a Hamiltonian $\hat{H}$ and an arbitrary state $\ket{\phi}$
both in some Hilbert space H. Because $\hat{H}$ is Hermitian, we are guaranteed
that it has a complete set of eigenvectors and eigenvalues, ${\ket{\psi}_i}$ and 
${E_i}$. Suppose that the lowest eigenstate has some, perhaps unknown, 
eigenvalue $E_0$. The Variational Principle is that the expectation value of 
$\hat{H}$ in the state $\ket{\phi}$ is
\begin{equation}
    \frac{\bra{\phi}\hat{H}\ket{\phi}}{\braket{\phi}{\phi}} \geq E_0.
\end{equation}
This is straightforward to prove since we can always write $\ket{\phi} = \sum_ic_i\ket{\psi}_i$.
Then equation (1) can be written
\begin{equation}\begin{split}
    \frac{\bra{\phi}\hat{H}\ket{\phi}}{\braket{\phi}{\phi}} &= 
    \frac{\sum_jc^*_j\bra{\psi}_j\hat{H}\sum_ic_i\ket{\psi}_i}{\sum_jc^*_j\bra{\psi}_j\sum_ic_i\ket{\psi}_i} \\
    &= \frac{\sum_{j,i}c^*_jc_i\bra{\psi}_j\hat{H}\ket{\psi}_i}{\sum_{j,i}c^*_jc_i\delta_{i,j}} \\
    &= \frac{1}{\sum_{i}|c_i|^2}\sum_{j,i}c^*_jc_iE_i\delta_{j,i} \\
    &= \frac{1}{\sum_{i}|c_i|^2}\sum_{j}|c_j|^2E_j
\end{split}\end{equation}
Since $E_0\leq E_i$ for all $i$, each term satisfies $|c_j|^2E_0\leq|c_j|^2E_j$ and
\begin{equation}
    \frac{1}{\sum_{i}|c_i|^2}\sum_{j}|c_j|^2E_j \geq \frac{1}{\sum_{i}|c_i|^2}E_0\sum_{j}|c_j|^2 \geq E_0.
\end{equation}
This inequality allows us to propose a model wavefunction and have the guarantee 
that our ground state will be bounded.




\chapter{Sample Density Matrix Files}
Here I have provided some redacted density matrix files to provide some context
for the discussion given on reading in density matrices. The following are data
from input files to be read into PNISM. The format of the density matrix files is
\begin{equation}
    \bra{Final\ State}[\pi^\dagger_a \pi_c]_{J_t}\ket{Initial\ State}
\end{equation}

These particle density matrix files are
used for calculating $^{20}$Ne in the sd model space with an inert $^{16}$O core. This means
that there are two valence protons and two valence neutrons. This first file
contains the excitation spectra and density matrices for a nuclei with
two protons and zero neutrons with $M=J_z=0$.
\begin{verbatim}
  BIGSTICK Version 7.8.1 Sept 2017
  single-particle file = sd
           2           0    <--- #protons, #neutrons
           0 +              <--- Jz, parity
  Time to compute jumps :    9.9999993108212948E-004
  Time to compute jumps :    0.0000000000000000     
  State      E        Ex         J      T 
    1    -11.77906   0.00000    -0.000   1.000
    2     -9.84945   1.92961     2.000   1.000
    3     -8.37591   3.40315     4.000   1.000
...
 Initial state #    2 E =   -9.84945 2xJ, 2xT =    4   2
 Final state   #    4 E =   -7.56341 2xJ, 2xT =    4   2
 Jt =   0, proton      neutron 
    1    1  -0.01892   0.00000
    2    2   0.45231   0.00000
    3    3  -0.75667   0.00000
 Jt =   2, proton      neutron 
    1    1  -0.01948   0.00000
    1    2   0.04273   0.00000
    1    3  -0.00230   0.00000
    2    1  -0.05930   0.00000
    2    2  -0.65142   0.00000
    2    3   0.40773   0.00000
    3    1  -0.03520   0.00000
    3    2  -0.71134   0.00000
 Jt =   4, proton      neutron 
    1    2   0.06297   0.00000
    2    1  -0.13163   0.00000
    2    2   0.08321   0.00000
...
\end{verbatim}
This second files contains the excitation spectra and density matrices for a nuclei with
two protons and zero neutrons with $M=J_z=1$. Notice that the states number labels
do not match up between the two files. The $M=J_z=1$ results are missing states
with $J=0$; you cannot have total angular momentum less than the z-component of
the angular momentum. Also notice that density matrix elements between the same
states (identified by $E$, $J$, and $T$ quantum numbers) are identical up to a 
phase. The $M=J_z=1$ density matrix file also has the entire set of $J_t$ quantum
numbers for a given initial and final state while the $M=J_z=1$ density matrix 
file is missing odd $J_t$ quantum numbers. Both solutions are required by PNISM
to obtain the entire density matrix.
\begin{verbatim}
  BIGSTICK Version 7.8.1 Sept 2017
  single-particle file = sd
           2           0
           2 +
  Time to compute jumps :    1.0000001639127731E-003
  Time to compute jumps :    0.0000000000000000     
  State      E        Ex         J      T 
    1     -9.84945   0.00000     2.000   1.000
    2     -8.37591   1.47354     4.000   1.000
    3     -7.56341   2.28605     2.000   1.000
...
Initial state #    1 E =   -9.84945 2xJ, 2xT =    4   2
 Final state   #    3 E =   -7.56341 2xJ, 2xT =    4   2
 Jt =   0, proton      neutron 
    1    1   0.01892   0.00000
    2    2  -0.45231   0.00000
    3    3   0.75667   0.00000
 Jt =   1, proton      neutron 
    1    1   0.01886   0.00000
    1    2  -0.03956   0.00000
    1    3   0.02182   0.00000
    2    1   0.00724   0.00000
    2    2   0.04334   0.00000
    3    1  -0.03337   0.00000
    3    3  -0.31605   0.00000
 Jt =   2, proton      neutron 
    1    1   0.01948   0.00000
    1    2  -0.04273   0.00000
    1    3   0.00230   0.00000
    2    1   0.05930   0.00000
    2    2   0.65142   0.00000
    2    3  -0.40773   0.00000
    3    1   0.03520   0.00000
    3    2   0.71134   0.00000
 Jt =   3, proton      neutron 
    1    1   0.02028   0.00000
    1    2  -0.03333   0.00000
    2    1   0.13226   0.00000
    2    2   0.75618   0.00000
    2    3   0.17724   0.00000
    3    2  -0.29377   0.00000
 Jt =   4, proton      neutron 
    1    2  -0.06297   0.00000
    2    1   0.13163   0.00000
    2    2  -0.08321   0.00000
...
\end{verbatim}












